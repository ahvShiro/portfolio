\documentclass[10pt, letterpaper]{article}
\usepackage[T1]{fontenc}
\usepackage[brazil]{babel}
% Packages:
\usepackage[
    ignoreheadfoot, % set margins without considering header and footer
    top=1 cm, % seperation between body and page edge from the top
    bottom=2 cm, % seperation between body and page edge from the bottom
    left=1 cm, % seperation between body and page edge from the left
    right=1 cm, % seperation between body and page edge from the right
    footskip=1.0 cm, % seperation between body and footer
    % showframe % for debugging 
]{geometry} % for adjusting page geometry
\usepackage{titlesec} % for customizing section titles
\usepackage{tabularx} % for making tables with fixed width columns
\usepackage{array} % tabularx requires this
\usepackage[dvipsnames]{xcolor} % for coloring text
\definecolor{primaryColor}{rgb}{0.56, 0.0, 1.0}% define primary color
\usepackage{enumitem} % for customizing lists
\usepackage{fontawesome5} % for using icons
\usepackage{amsmath} % for math
\usepackage[
    pdftitle={Currículo - Arthur Shiroshima},
    pdfauthor={Arthur Shiroshima},
    pdfcreator={LaTeX with RenderCV},
    colorlinks=true,
    urlcolor=primaryColor
]{hyperref} % for links, metadata and bookmarks
\usepackage[pscoord]{eso-pic} % for floating text on the page
\usepackage{calc} % for calculating lengths
\usepackage{bookmark} % for bookmarks
\usepackage{lastpage} % for getting the total number of pages
\usepackage{changepage} % for one column entries (adjustwidth environment)
\usepackage{paracol} % for two and three column entries
\usepackage{ifthen} % for conditional statements
\usepackage{needspace} % for avoiding page brake right after the section title
\usepackage{iftex} % check if engine is pdflatex, xetex or luatex

\usepackage{soul}
\setul{0.3ex}{0.1ex}
\setulcolor{primaryColor}   % sets the underline color


% Ensure that generate pdf is machine readable/ATS parsable:
\ifPDFTeX
    \input{glyphtounicode}
    \pdfgentounicode=1
    % \usepackage[T1]{fontenc} % this breaks sb2nov
    \usepackage[utf8]{inputenc}
    \usepackage{lmodern}
\fi



% Some settings:
\AtBeginEnvironment{adjustwidth}{\partopsep0pt} % remove space before adjustwidth environment
\pagestyle{empty} % no header or footer
\setcounter{secnumdepth}{0} % no section numbering
\setlength{\parindent}{0pt} % no indentation
\setlength{\topskip}{0pt} % no top skip
\setlength{\columnsep}{0cm} % set column seperation
% \makeatletter
% \let\ps@customFooterStyle\ps@plain % Copy the plain style to customFooterStyle
% \patchcmd{\ps@customFooterStyle}{\thepage}{
%     \color{gray}\textit{\small Arthur Shiroshima - Página \thepage{} de \pageref*{LastPage}}
% }{}{} % replace number by desired string
% \makeatother
% \pagestyle{customFooterStyle}

\titleformat{\section}{\needspace{4\baselineskip}\bfseries\large}{}{0pt}{}[\vspace{1pt}\titlerule]

\titlespacing{\section}{
    % left space:
    -1pt
}{
    % top space:
    0.3 cm
}{
    % bottom space:
    0.2 cm
} % section title spacing

\renewcommand\labelitemi{$\circ$} % custom bullet points
\newenvironment{highlights}{
    \begin{itemize}[
        topsep=0.10 cm,
        parsep=0.10 cm,
        partopsep=0pt,
        itemsep=0pt,
        leftmargin=0.4 cm + 10pt
    ]
}{
    \end{itemize}
} % new environment for highlights

\newenvironment{highlightsforbulletentries}{
    \begin{itemize}[
        topsep=0.10 cm,
        parsep=0.10 cm,
        partopsep=0pt,
        itemsep=0pt,
        leftmargin=10pt
    ]
}{
    \end{itemize}
} % new environment for highlights for bullet entries


\newenvironment{onecolentry}{
    \begin{adjustwidth}{
        0.2 cm + 0.00001 cm
    }{
        0.2 cm + 0.00001 cm
    }
}{
    \end{adjustwidth}
} % new environment for one column entries

\newenvironment{twocolentry}[2][]{
    \onecolentry
    \def\secondColumn{#2}
    \setcolumnwidth{\fill, 4.5 cm}
    \begin{paracol}{2}
}{
    \switchcolumn \raggedleft \secondColumn
    \end{paracol}
    \endonecolentry
} % new environment for two column entries

\newenvironment{header}{
    \setlength{\topsep}{0pt}\par\kern\topsep\centering\linespread{1.5}
}{
    \par\kern\topsep
} % new environment for the header

\newcommand{\placelastupdatedtext}{} % For some reason it breaks when I remove this

% save the original href command in a new command:
\let\hrefWithoutArrow\href

% new command for external links:
\renewcommand{\href}[2]{\hrefWithoutArrow{#1}{\ifthenelse{\equal{#2}{}}{ }{#2 }\raisebox{.15ex}{\footnotesize \faExternalLink*}}}


\begin{document}
    \newcommand{\AND}{\unskip
        \cleaders\copy\ANDbox\hskip\wd\ANDbox
        \ignorespaces
    }
    \newsavebox\ANDbox
    \sbox\ANDbox{}

    \placelastupdatedtext
    \begin{header}
        \textbf{\fontsize{24 pt}{24 pt}\selectfont Arthur Heiji Voos Shiroshima}

        \vspace{0.2 cm}

        \normalsize
        \mbox{{\color{black}\footnotesize\faMapMarker*}\hspace*{0.13cm}Paranavaí - PR}%
        \kern 0.25 cm%
        \AND%
        \kern 0.25 cm%
        \mbox{{\color{black}\footnotesize\faEnvelope[regular]}\hspace*{0.13cm}\hrefWithoutArrow{mailto:arthur.shiroshima@proton.me}{\textcolor{primaryColor}{\ul{arthur.shiroshima@proton.me}}}}%
        \kern 0.25 cm%
        \AND%
        \kern 0.25 cm\mbox{{\color{black}\footnotesize\faLink}\hspace*{0.13cm}\hrefWithoutArrow{https://shiroshima.com.br/}{\textcolor{primaryColor}{\ul{shiroshima.com.br}}}}%
        \kern 0.25 cm%
        \AND%
        \kern 0.25 cm%
        \mbox{{\color{black}\footnotesize\faLinkedinIn}\hspace*{0.13cm}\hrefWithoutArrow{https://linkedin.com/in/arthur-shiroshima}{\textcolor{primaryColor}{\ul{arthur-shiroshima}}}}%
        \AND%
        \kern 0.25 cm%
        \mbox{{\color{black}\footnotesize\faGithub}\hspace*{0.13cm}\hrefWithoutArrow{https://github.com/ahvshiro}{\textcolor{primaryColor}{\ul{ahvShiro}}}}%
    \end{header}


    \section{Objetivo}

        \begin{onecolentry}
            Sou estudante de Engenharia de Software no IFPR, com interesse em desenvolvimento back-end. Busco aplicar meus conhecimentos em lógica de programação, APIs e bancos de dados, contribuindo com soluções escaláveis e bem estruturadas.
        \end{onecolentry}

    % \section{Tópicos}

    % \begin{onecolentry}
    %     \begin{highlightsforbulletentries}


    %     \item Each section title is arbitrary and each section contains a list of entries.

    %     \item There are 7 unique entry types: \textit{BulletEntry}, \textit{TextEntry}, \textit{EducationEntry}, \textit{ExperienceEntry}, \textit{NormalEntry}, \textit{PublicationEntry}, and \textit{OneLineEntry}.

    %     \item Select a section title, pick an entry type, and start writing your section!

    %     \item \href{https://docs.rendercv.com/user_guide/}{Here}, you can find a comprehensive user guide for RenderCV.


    %     \end{highlightsforbulletentries}
    % \end{onecolentry}

    
    \section{Experiência}

        % % UOL %%%%%%%%%%%%%%%%%%%%%%%%%%%%%%%%%%%%%%%%%%%%%%%%%%%%%%%%%%%%%%%%%%%%%%%%%%%%%%%%%%%%%%%%%%%%%%%%%%%%%%%%%%%%%%%%%
        % \begin{twocolentry}{
        % \textit{Home Office}    
            
        % \textit{Mai 2025 – Atual}}
        %     \textbf{Estagiário em Backend Development}
            
        %     \textit{Compass UOL}
        % \end{twocolentry}

        % \vspace{0.10 cm}
        
        % \begin{onecolentry}
        %     \begin{highlights}
        %         \item To come...
        %     \end{highlights}
        % \end{onecolentry}
        % % UOL %%%%%%%%%%%%%%%%%%%%%%%%%%%%%%%%%%%%%%%%%%%%%%%%%%%%%%%%%%%%%%%%%%%%%%%%%%%%%%%%%%%%%%%%%%%%%%%%%%%%%%%%%%%%%%%%%
        
        % \vspace{0.2 cm}

        % SESAU %%%%%%%%%%%%%%%%%%%%%%%%%%%%%%%%%%%%%%%%%%%%%%%%%%%%%%%%%%%%%%%%%%%%%%%%%%%%%%%%%%%%%%%%%%%%%%%%%%%%%%%%%%%%%%%
        \begin{twocolentry}{
        \textit{Paranavaí, PR}    
            
        \textit{Jul 2024 – Mai 2025}}
            \textbf{Estagiário em Informática}
            
            \textit{Prefeitura Municipal de Paranavaí, Secretaria da Saúde}
        \end{twocolentry}

        \vspace{0.10 cm}
        
        \begin{onecolentry}
            \begin{highlights}
                \item Atendi de 20 a 25 chamados semanais via telefone e TightVNC, solucionando problemas de conectividade, hardware e sistema.
                \item Realizei suporte de hardware, software e redes presencial em unidades de saúde, com autonomia para diagnosticar falhas e aplicar correções imediatas.
                \item Participei da migração da UPA durante sua renovação, mapeando ramais da nova estrutura e auxiliando na instalação de dispositivos de rede como switches e pontos de acesso wireless.
                \item Desenvolvi habilidades de atendimento técnico, priorização de tarefas e comunicação com diferentes perfis de usuários.
            \end{highlights}
        \end{onecolentry}
        % SESAU %%%%%%%%%%%%%%%%%%%%%%%%%%%%%%%%%%%%%%%%%%%%%%%%%%%%%%%%%%%%%%%%%%%%%%%%%%%%%%%%%%%%%%%%%%%%%%%%%%%%%%%%%%%%%%%



    \section{Educação}

        \begin{twocolentry}{
                        
        \textit{Fev 2024 – Dez 2027}}
            \textbf{Instituto Federal do Paraná} - Campus Paranavaí

            \textit{Bacharelado em Engenharia de Software}
            
        \end{twocolentry}
    
    % \section{Projetos}

    %     \begin{twocolentry}{
                        
    %     \textit{\href{https://github.com/ahvShiro/website}{GitHub}}}
    %         \textbf{Website Pessoal}
    %     \end{twocolentry}

    %     \vspace{0.10 cm}
    %     \begin{onecolentry}
    %         \begin{highlights}
    %             \item Desenvolvi um 
    %             \item Ferramentas utilizadas: HTML, CSS, GitHub Pages, 
    %         \end{highlights}
    %     \end{onecolentry}


    %     \vspace{0.2 cm}

    %     \begin{twocolentry}{
            
    %     \textit{\href{https://github.com/sinaatalay/rendercv}{github.com/name/repo}}}
    %         \textbf{Synchronized Desktop Calendar}
    %     \end{twocolentry}

    %     \vspace{0.10 cm}
    %     \begin{onecolentry}
    %         \begin{highlights}
    %             \item 
    %             \item Ferramentas Usadas:
    %         \end{highlights}
    %     \end{onecolentry}

    
    \section{Habilidades}

        \begin{onecolentry}
            Java, Python, PHP, JavaScript, C\textbf{ | } Node.js, React Native, Expo\textbf{ | } PostgreSQL, SQLite, MariaDB, MongoDB\textbf{ | } Git, Docker, Apache, Postman
        \end{onecolentry}
    
    \section{Certificações}

        % CERTIPROF %%%%%%%%%%%%%%%%%%%%%%%%%%%%%%%%%%%%%%%%%%%%%%%%%%%%%%%%%%%%%%%%%%%%%%%%%%%%%%%%%%%%%%%%%%%%%%%%%%%%%%%%
        \begin{twocolentry}{        
        \textit{\href{https://drive.google.com/file/d/1clGQ68hbqWuvYoqf4kLtjp3EavD6oUl_/view}{Certiprof}}}
            \textbf{Scrum Foundation Professional Certificate (SFPC™)}
        \end{twocolentry}

        \vspace{0.10 cm}
        
        \begin{onecolentry}
            \begin{highlights}
                \item Fundamentos da metodologia Scrum segundo o \textit{Scrum Guide™ – The Definitive Guide to Scrum: The Rules of the Game}, por Ken Schwaber and Jeff Sutherland
            \end{highlights}
        \end{onecolentry}
        % CERTIPROF %%%%%%%%%%%%%%%%%%%%%%%%%%%%%%%%%%%%%%%%%%%%%%%%%%%%%%%%%%%%%%%%%%%%%%%%%%%%%%%%%%%%%%%%%%%%%%%%%%%%%%%%

        \vspace{0.2 cm}

        % INGLES INTERMEDIARIO II %%%%%%%%%%%%%%%%%%%%%%%%%%%%%%%%%%%%%%%%%%%%%%%%%%%%%%%%%%%%%%%%%%%%%%%%%%%%%%%%%%%%%%%%%%
        \begin{twocolentry}{            
        \textit{\href{https://drive.google.com/file/d/18BuqR67oopjdiHe33fjyTVAAJt3ZBsEi/view?usp=sharing}{CELIF - IFPR}}}
            \textbf{Curso de Inglês Intermediário II}
        \end{twocolentry}

        \vspace{0.10 cm}
        
        \begin{onecolentry}
            \begin{highlights}
                \item Desenvolvimento das habilidades de leitura, escrita, fala e compreensão, por meio de atividades escritas, conversações e avaliações orais e escritas
            \end{highlights}
        \end{onecolentry}
        % INGLES INTERMEDIARIO II %%%%%%%%%%%%%%%%%%%%%%%%%%%%%%%%%%%%%%%%%%%%%%%%%%%%%%%%%%%%%%%%%%%%%%%%%%%%%%%%%%%%%%%%%%

        \vspace{0.2 cm}

        % GIT WORKFLOW %%%%%%%%%%%%%%%%%%%%%%%%%%%%%%%%%%%%%%%%%%%%%%%%%%%%%%%%%%%%%%%%%%%%%%%%%%%%%%%%%%%%%%%%%%%%%%%%%%%%%
        \begin{twocolentry}{            
        \textit{\href{https://www.upinside.com.br/certificados/6734048202410}{Upinside Treinamentos}}}
            \textbf{Git Workflow}
        \end{twocolentry}

        \vspace{0.10 cm}
        \begin{onecolentry}
            \begin{highlights}
                \item Versionamento de projetos, repositórios no Github, trabalhando com git diff, branches e tags, trabalhando com IDE produtiva.
            \end{highlights}
        \end{onecolentry}
        % GIT WORKFLOW %%%%%%%%%%%%%%%%%%%%%%%%%%%%%%%%%%%%%%%%%%%%%%%%%%%%%%%%%%%%%%%%%%%%%%%%%%%%%%%%%%%%%%%%%%%%%%%%%%%%%

        \vspace{0.2 cm}

        % AMBIENTE LOCAL COM DOCKER %%%%%%%%%%%%%%%%%%%%%%%%%%%%%%%%%%%%%%%%%%%%%%%%%%%%%%%%%%%%%%%%%%%%%%%%%%%%%%%%%%%%%%%%
        \begin{twocolentry}{            
        \textit{\href{https://www.upinside.com.br/certificados/6734070202412}{Upinside Treinamentos}}}
            \textbf{Ambiente Local com Docker}
        \end{twocolentry}

        \vspace{0.10 cm}
        \begin{onecolentry}
            \begin{highlights}
                \item Instalação do Docker, criação e gestão de containers, imagens e volumes, incluindo automação com Dockerfile e docker-compose.
            \end{highlights}
        \end{onecolentry}
        % AMBIENTE LOCAL COM DOCKER %%%%%%%%%%%%%%%%%%%%%%%%%%%%%%%%%%%%%%%%%%%%%%%%%%%%%%%%%%%%%%%%%%%%%%%%%%%%%%%%%%%%%%%%
        
    \section{Idiomas}

        \begin{onecolentry}
            \textbf{Português:} Nativo
        \end{onecolentry}

        \vspace{0.1 cm}
        \begin{onecolentry}
            \textbf{Inglês:} Intermediário B1 – leitura, escrita, compreensão e conversação
        \end{onecolentry}

        \vspace{0.1 cm}
        \begin{onecolentry}
            \textbf{Japonês} Básico N5 – leitura e escrita
        \end{onecolentry}

        
\end{document}